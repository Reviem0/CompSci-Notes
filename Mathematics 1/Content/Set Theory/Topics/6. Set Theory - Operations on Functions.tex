% Options for packages loaded elsewhere
\PassOptionsToPackage{unicode}{hyperref}
\PassOptionsToPackage{hyphens}{url}
\documentclass[
]{article}
\usepackage{xcolor}
\usepackage{amsmath,amssymb}
\setcounter{secnumdepth}{-\maxdimen} % remove section numbering
\usepackage{iftex}
\ifPDFTeX
  \usepackage[T1]{fontenc}
  \usepackage[utf8]{inputenc}
  \usepackage{textcomp} % provide euro and other symbols
\else % if luatex or xetex
  \usepackage{unicode-math} % this also loads fontspec
  \defaultfontfeatures{Scale=MatchLowercase}
  \defaultfontfeatures[\rmfamily]{Ligatures=TeX,Scale=1}
\fi
\usepackage{lmodern}
\ifPDFTeX\else
  % xetex/luatex font selection
\fi
% Use upquote if available, for straight quotes in verbatim environments
\IfFileExists{upquote.sty}{\usepackage{upquote}}{}
\IfFileExists{microtype.sty}{% use microtype if available
  \usepackage[]{microtype}
  \UseMicrotypeSet[protrusion]{basicmath} % disable protrusion for tt fonts
}{}
\makeatletter
\@ifundefined{KOMAClassName}{% if non-KOMA class
  \IfFileExists{parskip.sty}{%
    \usepackage{parskip}
  }{% else
    \setlength{\parindent}{0pt}
    \setlength{\parskip}{6pt plus 2pt minus 1pt}}
}{% if KOMA class
  \KOMAoptions{parskip=half}}
\makeatother
\setlength{\emergencystretch}{3em} % prevent overfull lines
\providecommand{\tightlist}{%
  \setlength{\itemsep}{0pt}\setlength{\parskip}{0pt}}
\usepackage{bookmark}
\IfFileExists{xurl.sty}{\usepackage{xurl}}{} % add URL line breaks if available
\urlstyle{same}
\hypersetup{
  hidelinks,
  pdfcreator={LaTeX via pandoc}}

\author{}
\date{}

\begin{document}

\subsection{Composition}\label{composition}

If \(f: X \to Y\) and \(g:Y\to Z\) then \(f;g:X\to Z\) is a function
that is defined - \(f;g\) is sometimes written \(g \circ f\) -
\((f;g)(x) = g(f(x))\) -- function of \(f\) into \(g\) -
\((f \circ g)(x) = f(g(x))\) -- function of \(g\) into \(f\)

!{[}{[}Pasted image 20241102021337.png{]}{]} \#\# Associativity Function
composition is associative \((f;g);h = f;(g;h)\) Proof: 1. Setup - Let's
consider three functions \(f,g,h\) - Let \(x\) be any element in the
domain of \(h\) - We need to show that \(((f;g);h)(x) = (f;(g;h))(x)\)
2. Evaluate the left side: \(((f;g);h)(x)\) - By definition -
\((f;g)(y)~\text{means}~ g(f(y))\) - So \(((f;g);h)(x)\) means: -
\((f;g)(x) = g(f(x))\) - Then apply \(h\) to the result -
\((h(g(f(x))))\) 3. Evaluate the right side: - \((f;(g;h))(x)\) means: -
Apply \(f\) to \(x\): f(x) - Apply \((g;h)\) to the result -
\((h(g(f(x))))\) It is also true that \(id_{X};f = f = f;id_{Y}\)

\subsection{Inverting Functions}\label{inverting-functions}

Given a function \(f:X \to Y\), and inverse (if it exists) is a function
\(f^{-1}:Y\to X\) such that: - \(f^{-1}(f(x)) = x\) - \(f(f^{-1}(y))=y\)
This is the same as: \[
f;f^{-1} = id_{x}~~\text{and}~~f^{-1};f = id_{y}
\]

\end{document}
